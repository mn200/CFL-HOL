\section{Related work and conclusions}

In the field of language theory, Nipkow~\citep{nipkow98} provided a
verified and executable lexical analyzer generator. This work is the
closest in nature to the mechanisation we have done.

%%% rewrite this
Formalisation of proofs is an incredibly detailed process and as such
gets easily affected by numerous concerns that seem trivial in the
context of a text proof. Concerns about the extent of explicitness of
the proof steps, the structures for representing the proof problem,
finiteness of the data structures being used, termination of an
algorithm and executability in some cases are just a handful of issues
that result in the bloating of a mechanised proof.

These result in a number of what we call ``gap'' proofs. These
``gaps'' cover the deductive steps that get omitted in a textbook
proof and the intermediate results needed because of the particular
mechanisation technique. Formalisation of a theory results in tools,
techniques and an infrastructure that forms the basis of verifying
tools based on the theory for example parsers, compilers, etc. Working
in a well understood domain is useful in understanding the immense
deviations that automation usually results in.  More often than not
the techniques for dealing with a particular problem in a domain are
hard to generalise. The only solution in such cases is to have an
extensive library at one's call.

With this work, we have formalised a large portion of the background
theory of context-free grammars and pushdown automata. We hope this
theory can provide a platform for other works in the area of
context-free languages. This formalisation is also a good basis for
teaching language theory and for exploring the proof process that form
a central part of theorem proving, both textually and using an
automation tool. Automated theorem proving can have a beneficial
affect for teaching logic and mathematics courses as evident from the
use of Coqweb~\citep{coqweb}.  Coqweb provides a language that is
close to standard mathematical language. It also provides an interface
for solving exercises using the theorem prover Coq. Proofs are
essentially performed by clicking. Such web-based approaches that rope
in automated proof techniques are being used to teach subjects like
discrete mathematics and logic at the undergraduate
level. Blanc~\et~have used the Coqweb environment to explore how proof
assistants can help teachers to explain the concept of a proof and how
to search for one~\citep{pate07} and Benjamin Pierce to teach courses
at University of Pennsylvania~\citep{Pierce:LambdaTA-ITP}. Going
beyond teaching theorem proving as a course in its own right, Pierce
has used it as a framework to teach programming language
foundations~\citep{Pierce:LambdaTA}. The formalisation of language
theory, a standard undergraduate logic course, can be an excellent
domain for such endeavours.

Table~\ref{tab:numbers} summarises the proof effort. It took six
months to complete the formalisation.
% TABLE OF STATS
\begin{table}[!ht]
\begin{center}
\begin{tabular}{lrrr}
  \textbf{Theory}&\textbf{LOC}&\textbf{\#Definitions}&\textbf{\#Proofs}\\
  \hline
  CFGs - background       & 3680 & 36 & 189 \\
  PDA - background        & 1846 & 15 & 47 \\
  Empty Stack Acceptance $\iff$ Final State Acceptance & 1795 & 6 & 75 \\
  PDA == CFG & 2598 & 16 & 50\\
  Closure properties & 1686 & 12 & 91\\
\end{tabular}
\end{center}
\caption{Summary of proof effort}
\label{tab:numbers}
\end{table}

HOL sources for the work are available at \url{http://github.com/mn200/CFL-HOL}.

\paragraph{Acknowledgements}
NICTA is funded by the Australian Government as represented by the
Department of Broadband, Communications and the Digital Economy and
the Australian Research Council through the ICT Centre of Excellence
program.


%%% Local Variables:
%%% mode: latex
%%% TeX-master: "pda"
%%% End:
