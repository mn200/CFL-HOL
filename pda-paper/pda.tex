\documentclass[1p,review,authoryear]{elsarticle}
\usepackage{latexsym}
\usepackage{amssymb}
\usepackage{amsmath}
\usepackage{alltt}
\usepackage{times}
\usepackage{graphicx}
\usepackage{underscore}
\usepackage{holtexbasic}
%\usepackage{verbatim}
\usepackage{pstricks}
\usepackage{url}
\usepackage{tikz}
\usepackage{hyperref}

\newcommand{\mcite}[1]{(\citet{#1})}
\newcommand{\et}{\textit{et al}}
\newcommand{\around}{\raisebox{.3ex}{\tiny$\sim$}}
\newtheorem{mythm}{Theorem}[section]
\newtheorem{holthm}{HOL Theorem}
\newtheorem{hollemma}{HOL Lemma}
\newtheorem{holdef}{HOL Definition}
\newtheorem{holgap}{Gap Theorem}[section]
\newcommand{\V}{\forall}
\newcommand{\E}{\exists}
\newcommand{\ie}{\textit{i.e.}}
\newcommand{\eg}{\textit{e.g.}}
\newcommand{\derive}{\Rightarrow}
\newcommand{\lderive}{\stackrel{l}{\derive}}
\newcommand{\rderive}{\stackrel{r}{\derive}}
\newcommand{\derives}[1]{\Rightarrow^{#1}}
\newcommand{\rderives}[1]{{\stackrel{r}{\derive}}{}^{#1}}
\newcommand{\lderives}[1]{{\stackrel{l}{\derive}}{}^{#1}}
\newenvironment{salltt}{\begin{samepage}\small\begin{alltt}}{\end{alltt}\end{samepage}}

\newenvironment{gaps}
{% This is the begin code
\noindent\par\textbf{Filling in the gaps}\begin{it}\par\noindent
}
{% This is the end code
\end{it} }
\newtheorem{theorem}{Theorem}[section]
\newtheorem{lemma}[theorem]{Lemma}
\newtheorem{proposition}[theorem]{Proposition}
\newtheorem{corollary}[theorem]{Corollary}
\newenvironment{proof}[1][Proof]{\begin{trivlist}
\item[\hskip \labelsep {\bfseries #1}]}{\end{trivlist}}
\newenvironment{definition}[1][Definition]{\begin{trivlist}
\item[\hskip \labelsep {\bfseries #1}]}{\end{trivlist}}



\begin{document}
\begin{frontmatter}
\date{}

\title{A Mechanisation of the Theory of Context-Free Languages in HOL4}

\author[ab]{Aditi Barthwal\corref{cor1}}
\ead{Aditi.Barthwal@anu.edu.au}
\author[mn]{ Michael Norrish\corref{cor2}}
\ead{Michael.Norrish@nicta.com.au}
\address[ab]{Australian National University}
\address[mn]{Canberra Research Lab., NICTA}
\cortext[cor1]{Principal corresponding author}
\cortext[cor2]{Corresponding author}



\begin{abstract}
We describe the mechanisation of some foundational results in the theory of context-free languages~(CFLs), using the HOL4 system.
We first show that two standard models, the pushdown automata~(PDAs) and context-free grammars~(CFGs), actually coincide.
With both models to hand, we can then show a number of basic, but important results.
For example, we prove the basic closure properties of the context-free languages such as union and concatenation.
Using PDAs, we are able to mechanise an elegant proof of closure under inverse homomorphism.
We also discuss the varying extent to which textbook proofs (we follow Hopcroft and Ullman) and our mechanisations diverge: sometimes elegant textbook proofs remain elegant in HOL; sometimes the required mechanisation effort blows up unconscionably.
\end{abstract}

\begin{keyword}
context-free languages \sep context-free grammars \sep pushdown automata \sep closure properties \sep HOL4
\end{keyword}
\end{frontmatter}

\input{introduction}
\input{cfg}
\input{pdaDef}
\input{laeslafs}
\input{cfg2pda}
\input{pda2cfg}
\input{cfls}
\section{Related work}

In the general area of mechanised language theory, the earliest work we are aware of is by \citeauthor{nipkow98}~(\citeyear{nipkow98}).
That paper describes a verified and executable lexical analyzer generator.
%
Focusing on core language theory as it does, we feel this work is the closest in nature our own mechanisation, though of course it covers regular rather than context-free languages.

Apart from our own earlier work on SLR parsing, there have recently been a number of papers directly concerned with mechanisation and verification of specific approaches to parsing.
For example, \cite{koprowski11:trx} describe the construction of a verified parser for \emph{expression grammars}.
This formalism allows for parsers for a large class of languages to be generated from natural specifications that look a great deal like context-free grammars.
Another example is \cite{ridge2011:cfg-parsing}, which work presents a verified parser for all possible context-free grammars, using an admirably simple algorithm.
The drawback is that, as presented, the algorithm is of complexity~$O(n^5)$.

Parser combinators, popular in functional programming languages, are another approach to the general parsing problem, and there has been some mechanisation work in this area.
For example, \citet{Danielsson2010:TPC} presents a library of parser combinators that have been verified (in the Agda system) to guarantee termination of parsing.


\section{Conclusions}


%%% rewrite this
Formalisation of proofs is an incredibly detailed process and as such
gets easily affected by numerous concerns that seem trivial in the
context of a text proof. Concerns about the extent of explicitness of
the proof steps, the structures for representing the proof problem,
finiteness of the data structures being used, termination of an
algorithm and executability in some cases are just a handful of issues
that result in the bloating of a mechanised proof.

These result in a number of what we call ``gap'' proofs. These
``gaps'' cover the deductive steps that get omitted in a textbook
proof and the intermediate results needed because of the particular
mechanisation technique. Formalisation of a theory results in tools,
techniques and an infrastructure that forms the basis of verifying
tools based on the theory for example parsers, compilers, etc. Working
in a well understood domain is useful in understanding the immense
deviations that automation usually results in.  More often than not
the techniques for dealing with a particular problem in a domain are
hard to generalise. The only solution in such cases is to have an
extensive library at one's call.

With this work, we have formalised a large portion of the background
theory of context-free grammars and pushdown automata. We hope this
theory can provide a platform for other works in the area of
context-free languages. This formalisation is also a good basis for
teaching language theory and for exploring the proof process that form
a central part of theorem proving, both textually and using an
automation tool. Automated theorem proving can have a beneficial
affect for teaching logic and mathematics courses as evident from the
use of Coqweb~\citep{coqweb}.  Coqweb provides a language that is
close to standard mathematical language. It also provides an interface
for solving exercises using the theorem prover Coq. Proofs are
essentially performed by clicking. Such web-based approaches that rope
in automated proof techniques are being used to teach subjects like
discrete mathematics and logic at the undergraduate
level. Blanc~\et~have used the Coqweb environment to explore how proof
assistants can help teachers to explain the concept of a proof and how
to search for one~\citep{pate07} and Benjamin Pierce to teach courses
at University of Pennsylvania~\citep{Pierce:LambdaTA-ITP}. Going
beyond teaching theorem proving as a course in its own right, Pierce
has used it as a framework to teach programming language
foundations~\citep{Pierce:LambdaTA}. The formalisation of language
theory, a standard undergraduate logic course, can be an excellent
domain for such endeavours.

Table~\ref{tab:numbers} summarises the proof effort. It took six
months to complete the formalisation.
% TABLE OF STATS
\begin{table}[!ht]
\begin{center}
\begin{tabular}{lrrr}
  \textbf{Theory}&\textbf{LOC}&\textbf{\#Definitions}&\textbf{\#Proofs}\\
  \hline
  CFGs---background       & 3680 & 36 & 189 \\
  PDAs---background        & 1846 & 15 & 47 \\
  Empty Stack Acceptance $\iff$ Final State Acceptance & 1795 & 6 & 75 \\
  PDA $\iff$ CFG & 2598 & 16 & 50\\
  Closure properties & 1686 & 12 & 91\\
\end{tabular}
\end{center}
\caption{Summary of proof effort}
\label{tab:numbers}
\end{table}

HOL sources for the work are available at \url{http://github.com/mn200/CFL-HOL}.

\paragraph{Acknowledgements}
NICTA is funded by the Australian Government as represented by the
Department of Broadband, Communications and the Digital Economy and
the Australian Research Council through the ICT Centre of Excellence
program.


%%% Local Variables:
%%% mode: latex
%%% TeX-master: "pda"
%%% End:

% LocalWords:  CFG combinators


\bibliographystyle {elsarticle-harv}
\bibliography{theoremProving}


\end{document}


% LocalWords:  PDAs CFGs CFLs
